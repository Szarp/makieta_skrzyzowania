\subsubsection{ O co chodzi}
W tym rozdziale czytelnik dowie się jak zacząć przygodę z programowaniem jeśli nigdy nie miał styczności z tym. Po wybraniu swojego poziomu są umieszczone tabelki z zadaniami oraz z listą zakupów .
\subsubsection{Podstawy podstaw - poziom 1}
Jest to idealne połączenie dla czytelników, którzy zaczynają przygodę z elektroniką oraz z programowaniem jednocześnie. Dzięki temu nauczy się podstaw z tych dziedzin. 
To, co czytelnik ma zrobić aby posiąść tę wiedzę będzie omówione w dalszej części dokumentu. Koszyk odpowiedni w na tym poziomie znajdziemy w tab: Zawartość koszyka na poszczególnych poziomach.
Poziom 1:
-któreś Arduino
-płytka stykowa
-kabelki (damsko-męskie i męsko-męskie)
czujnik, oporniki (do diod)
\subsubsection{Advanced beginner - poziom 2}
Czytelnik:
-zna podstawową składnię używaną w Arduino,
-zna różnice między Arduino Nano oraz Arduino Uno,
-potrafi uruchomić diodę wciskając przycisk,
-spędził co najmniej 5h w środowisku Arduino

Gdy czytelnik posiądzie wiedzę minimalną, może przejść do kolejnego poziomu wtajemniczenia w elektronikę, prócz poszerzenia naszego warsztatu, zmiany trudności zadań, czytelnik musi wykazać się swoją kreatywnością, ponieważ w tym momencie pojawiają się schody.

\subsubsection{Expert - poziom 3}
Jako, że jest to poradnik dla początkujących i średnio zaawansowanych, wyróżnimy tylko 3 poziomy wtajemniczenia. W tym poziomie nie ma już stricte list zakupów, ponieważ Expert sam dobrze wie co chce robić. 
\subsubsection{Lista zakupów}
Jak już wcześniej zostało napisane, w tym rozdziale zostanie poruszony temat listy zakupów oraz jak kupować

 Najlepszą stroną do kupowania elektroniki jest allegro, wszystkie dostępne (mniej lub bardziej podstawowe) czujniki znajdziemy w kategorii 'Arduino' (opcjonalnie nazwa czujnika np. magnetometr).

Jeżeli mamy czas i potrzebujemy więcej czujników, najlepiej zamówić je w Chinach poprzez Aliexpress. Nie należy się obawiać, że nie przyjdą jednak musimy uzbroić się w cierpliwość(najdłużej czekałem ok 70 dni).

%tabelka
\subsubsection{Zadania}
Tak jak w przypadku zakupów, problemy zadaniowe mają swoje własne serwisy, które ułatwiają pracę. Jest kilka ważnych zasad związanych z tym: szukamy po angielsku, problem wpisujemy w google i szukamy na Stackoverflow, do Arduino polecam forum Arduino też (przykładowe problemy)
\subsubsection{Reszta śmieci do zedytowania}
Zadania mają charakter edukacyjny i wszystkie uczą różnych knifów w programowaniu Arduino. Czasami jednak czytelnik będzie musiał wykazać się umiejętnością szperania w sieci. Podpowiedzi do zadań będą umieszczone w internecie.
%github, pozdrawaim
Jak zacząć:
Gdy kupiliśmy Arduino Nano wpinamy je jak na zdjęciu (zdj płytka stykowa-Arduino Nano) teraz możemy bez problemu wpiąć inne elementy (zdj serwa i np naszego sygnalizatora)
Gdy kupimy Arduino uno nie potrzebujemy nawet płytki stykowej ponieważ jest ono tak skonstruowane że możemy od razu wpinać urządzenie w płytkę
(znowu zdjęcia ale z uno)
To jest tak jakby wersja testowa. Na płytce stykowej łączymy wszytko kabelkami.
Załącznik lista zakupów: lv1, lv2, lv3

