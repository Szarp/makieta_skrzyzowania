
W tym rozdziale czytelnik dowie się jak zacząć przygodę z elektroniką jeśli nigdy nie miał styczności z tym. Rozdział ten będzie miał
nietypową budowę, ponieważ będzie podzielony na poziomy ze względu na stopień zaawansowania. Po wybraniu swojego poziomu są umieszczone tabelki z zadaniami oraz z listą zakupów .
\subsection{Podstawy podstaw -- poziom 1}
Jest to idealne połączenie dla czytelników, którzy zaczynają przygodę z~elektroniką oraz z~programowaniem jednocześnie. Dzięki temu nauczy się podstaw z~tych dziedzin. 
To, co czytelnik ma zrobić aby posiąść tę wiedzę będzie omówione w dalszej części dokumentu.
%Poziom 1:
%-Arduino Uno lub Nano,
%płytka stykowa,
%-kabelki (damsko-męskie i męsko-męskie),
%-czujnik, oporniki (do diod)
\subsection{Advanced beginner -- poziom 2}
Czytelnik:
\begin{itemize}
\item zna podstawową składnię używaną w Arduino
\item zna różnice między Arduino Nano oraz Arduino Uno
\item potrafi uruchomić diodę wciskając przycisk
\item spędził co najmniej 5h w środowisku Arduino
\end{itemize}

Gdy czytelnik posiądzie wiedzę minimalną, może przejść do kolejnego poziomu, prócz poszerzenia naszego warsztatu, zmiany trudności zadań, czytelnik musi wykazać się swoją kreatywnością, ponieważ w tym momencie pojawiają się schody.

\subsection{Expert -- poziom 3}
Jako, że jest to poradnik dla początkujących i średnio zaawansowanych, wyróżnimy tylko 3 poziomy. W tym poziomie nie ma już list zakupów, ponieważ Expert sam dobrze wie co chce robić.
\subsection{Lista zakupów i zadania}
Jak już wcześniej zostało napisane, w tym rozdziale zostanie poruszony temat listy zakupów oraz jak kupować.
%Zakupy:

Poziom pierwszy:
\begin{itemize}
\item Arduino Uno lub Nano
\item diody świecące (klor dowolny)
\item oporniki $300\Omega$
\item przyciski (opcjonalne)
\item termometr jako jeden czujnik
 \end{itemize}

Poziom drugi:
\begin{itemize}
\item czujniki dostępne na Allegro
\item kolejne Arduino (Nano najlepiej)
\item wyświetlacz
 \end{itemize}

Poziom trzeci:
\begin{itemize}
\item radia
\item siliniki (np. serwo)
\item batrie do Arduino (pilot bezprezwodowy)
\end{itemize}
        
Kolejnośc zadań jest od najprostrzego do najtrudniejszego:
 \begin{itemize}
\item mrugająca dioda (pin 13)
\item mrugająca dioda (zewnętrzna + rezystor)
\item świecąca dioda przy naciśnieciu przycisku (opornik wewnętrzny)
\item podpięcie czujnika (sterowanie przy pomocy I2C)
\item zaplaenie lampki przy wykryciu ruchu (łaczenie kilku czujników - jeden z~przykładów)
\item światło sterowane pilotem (2 Arduino i 2 radia)
\item sterowanie silnikami przy pomocy radia
 \end{itemize}
